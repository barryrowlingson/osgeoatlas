
\documentclass[a4paper,sfsidenotes,justified]{tufte-book}

\usepackage{pdfpages}
\usepackage{index}
\newcommand{\openepigraph}[2]{%
  %\sffamily\fontsize{14}{16}\selectfont
  \begin{fullwidth}
  \sffamily\LARGE
  \begin{doublespace}
  \noindent\allcaps{#1}\\% epigraph
  \noindent\large\allcaps{#2}% author
  \end{doublespace}
  \end{fullwidth}
}

\newcommand{\map}[1]{
\includepdf{#1/map.pdf}
\newpage
\begin{marginfigure}
\includegraphics[width=1in]{#1/author.jpg}
\end{marginfigure}

\section{A Visualisation}

This is just to show how space-time data could be presented
as a number of frames. This is a 2x2 layout, but you could
go up to 4x6 if you can still make out the data.

I think this is crime data in police sub-regions of Cumbria, England.

Make sure all the panels line up exactly, unlike in this example.

\index{Animation}
\index{R}

}

\title{The OSGeo Atlas}
\author{Barry S. Rowlingson, Editor}
\publisher{An OSGeo.org Production}

\makeindex

\begin{document}

\maketitle

\newpage
\begin{fullwidth}
~\vfill
\setlength{\parindent}{0pt}
\setlength{\parskip}{\baselineskip}
Copyright \copyright\ \the\year\ \thanklessauthor


\end{fullwidth}

\newpage\thispagestyle{empty}

\openepigraph{Where is the knowledge we have lost in information?}
{T. S. Eliot}

\openepigraph{For the execution of the voyage to the Indies, I did not make use of intelligence, mathematics or maps.}
{Christopher Columbus}

\tableofcontents

\chapter{Introduction}

Welcome to the OSGeo Atlas. The purpose of this book is to show how
beautiful cartography, insightful visualisation, and stunning design
can be produced with the help of open-source tools.

This book is itself produced using \LaTeX, the document typesetting system.

Each map figure appears on the left page with author notes on the right page.

\chapter{The Maps}
Some blurb about the maps here.

\map{Map/Ritalin}

\map{Map/Nuke}

\map{Map/Viz}

\backmatter

\printindex

\end{document}
