
\documentclass[a4paper,sfsidenotes,justified]{tufte-book}

\usepackage{pdfpages}
\usepackage{index}
\newcommand{\openepigraph}[2]{%
  %\sffamily\fontsize{14}{16}\selectfont
  \begin{fullwidth}
  \sffamily\LARGE
  \begin{doublespace}
  \noindent\allcaps{#1}\\% epigraph
  \noindent\large\allcaps{#2}% author
  \end{doublespace}
  \end{fullwidth}
}

\newcommand{\map}[1]{
\includepdf{#1/map.pdf}
\newpage
\begin{marginfigure}
\includegraphics[width=1in]{#1/author.jpg}
\end{marginfigure}

\includepdf{Map/Ritalin/map.pdf}
\section{Merseyside Ritalin}

\begin{marginfigure}
\includegraphics[width=2in,trim=7cm 7cm 7cm 15cm,clip=true]{Map/Ritalin/map.pdf}
\end{marginfigure}

Barry Rowlingson, Lancaster University

This map shows how the rate of Methylphenidate (Ritalin) prescription varies
greatly across the River Mersey in North-west England.

The data comes from monthly published costings and item counts of
prescription drugs for every general medical practice and clinic in 
England. This information is public open data, downloadable as a 
zipped CSV file.

Each month is loaded into a PostGIS database for retrieval. In this map,
we see point locations of medical practices coloured by spending per child.
A background grid is an interpolation computed using the R statistics system,
and significantly high regions are coloured bright red.

Maps such as these enable the exploration and investigation of disparities in
prescribing costs which may be due to underlying health reasons or changing
practices within the health service.

Software used:

\begin{itemize}
\item PostGIS \index{PostGIS}
\item R \index{R}
\item Quantum GIS \index{Quantum GIS}
\end{itemize}

Background map tiles are from OpenStreetMap. \index{OpenStreetMap}


}

\title{The OSGeo Atlas}
\author{Barry S. Rowlingson, Editor}
\publisher{An OSGeo.org Production}

\makeindex

\begin{document}

\maketitle

\newpage
\begin{fullwidth}
~\vfill
\setlength{\parindent}{0pt}
\setlength{\parskip}{\baselineskip}
Copyright \copyright\ \the\year\ \thanklessauthor


\end{fullwidth}

\newpage\thispagestyle{empty}

\openepigraph{Where is the knowledge we have lost in information?}
{T. S. Eliot}

\openepigraph{For the execution of the voyage to the Indies, I did not make use of intelligence, mathematics or maps.}
{Christopher Columbus}

\tableofcontents

\chapter{Introduction}

Welcome to the OSGeo Atlas. The purpose of this book is to show how
beautiful cartography, insightful visualisation, and stunning design
can be produced with the help of open-source tools.

This book is itself produced using \LaTeX, the document typesetting system.

Each map figure appears on the left page with author notes on the right page.

\chapter{The Maps}
Some blurb about the maps here.

\map{Map/Ritalin}

\map{Map/Nuke}

\map{Map/Viz}

\backmatter

\printindex

\end{document}
